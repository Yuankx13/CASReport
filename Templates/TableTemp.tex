\reposec{如何插入表格}
表格是一个很复杂的东西,latex能插入的表格总的来说是比较简单的,很难实现复杂的效果,不建议拿LaTeX当MS Office Excel使用。我这里只提供一个简单的例子。最好还是参考文件夹里附送的《102分钟掌握LaTeX》的文档来搭建表格。
\vspace{0.3cm}
\renewcommand\arraystretch{1.7}
\begin{center}
\begin{tabular}{>{\raggedleft\arraybackslash}p{110pt}|>{\raggedright\arraybackslash}p{165pt}}
	\hline
	2022年2月至5月&完成非晶磁通门原型机搭建\\
	\hline
	2022年4月至9月&继续探索进一步噪声压制手段\\
	\hline
	2022年6月至7月&形成成果并发表\\
	\hline
	2022年1月前&完成毕业论文和学位答辩\\
	\hline
\end{tabular}
\end{center}
\vspace{0.3cm}\par
解释一下上方的各个语句,请结合源代码查看。\par
$\backslash vspace\{\}$命令是垂直空白距离,即Verticle Space。\par
$\backslash renewcommand\backslash arraystretch\{\}$命令改动了array宏包默认的表格的行高。默认为1倍行距。花括号内填充的数字即为行距的倍数。\par
$\backslash begin\{center\}$居中环境开始,与后方的$\backslash end\{center\}$(居中环境结束)成对出现,作用是将表格整体居中。\par
$\backslash begin\{tabular\}$表格环境开始。接下来的花括号内都是表格的格式和修饰项。具体还是应该去看教程。我在这里也简单讲一下。
\begin{enumerate}
\item $>$是arrary宏包提供的功能,用于在表格定义参数的这个花括号内,直接对每一列的属性进行设置,避免单独设置的麻烦。
\item 第一列:
\begin{enumerate}
\item $\backslash raggedleft$,直译为左边参差不齐,也即是文字居右。
\item $\backslash arraybackslash$,使用了前一个标志$\backslash raggedleft$之后,表格的换行符$\backslash\backslash$可能会出错或失效,因此要加上这个标志,防止编译出错。
\item $p\{110pt\}$表格格式声明符,注意最前面并没有反斜线。这一个标志的意思是,表格为固定列宽度(p)110磅(pt)。除了p之外,还有\textit{l, c, r}可以用,分别代表居左,居中,居右。但是注意,后三种参数不能强制规定列宽,也不能用$\backslash raggedleft$, $\backslash arraybackslash$这两个命令,列宽将会自动适应。我还是推荐使用$p\{\}$格式的表格,手动调节,会更美观一些。但一些较大的表格用p格式就可能太繁琐了。
\end{enumerate}
\item $|$竖线代表表格第一列和第二列之间的分界线为实线。如果不写这个竖线,则这两列之间没有分界线。但其他设置不变。同理,你可以看到由于第一列之前、最后一列之后没有写这个符号,所以表格最左边和最右边都没有边框线。
\item $>$同上。
\item $\backslash raggedright$,同理,直译为右边参差不齐,也即是文字居左。除此之外,你也可以用$\backslash centering$来居中。
\item $\backslash arraybackslash$,同上。
\item $p\{165pt\}$同上,但固定列宽为165磅。
\end{enumerate}\par
表格搭建完成,来看看内部的各个命令。\par
$\backslash hline$,即\textit{horizontal line},水平线。可以看作是行间分界线。\par
$\&$,对齐符号,在表格环境里就是每列之间分界的位置。\par
$\backslash\backslash$换行符号。\par
想要深入了解,去看教程。\par

