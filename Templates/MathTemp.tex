\reposec{如何插入公式}
插入公式有三类办法。
\reposubsec{单行公式}
顾名思义,单行公式只适用于仅有一行的公式。单行公式会整体居中,并在页面最右形成编号,例如:
\begin{equation}
1+1=2\label{eqn01}
\end{equation}
但也可以通过给单行公式加*号来取消编号。
\begin{equation*}
1+1=3
\end{equation*}
也可以自定义编号
\begin{equation}
1+1=11\tag{99}
\end{equation}
公式编号可以被交叉引用$\backslash ref$命令查询到,但前提是先在对应的想要引用的公式后面加上$\backslash label$命令进行标注,例如式\ref{eqn01}中的写法。请参考源代码。但一切的交叉引用若要生效,xelatex编译命令至少执行两次。否则会显示两个问号。\par
同时,如果是方程组,虽然有多行,但应当只使用一个标号的情况,可以采用cases环境来编写,例如:
\begin{equation}
\begin{cases}
B_S &\to\ 0.2\sim0.3T,\\
W_H&\to\ 0,\\
\dfrac{B_R}{B_S}&\to\ 0.5;\\
H_C&\to\ 0,\\
\mu_{init}\ &\to\  max\ .
\end{cases}
\end{equation}
\reposubsec{多行公式}
多行公式与单行公式大体相同,上面说到的关于编号、cases等的用法也可以大致通用。只是,多行公式涉及一个对齐的问题。多行公式的每行之间以$\&$这个符号的位置对齐,这个符号并不显示。在对其的前提下,整体再居中。例如下面的几行,以等于号对齐。其实,在上面的cases环境中,我们已经用到了$\&$来帮助对齐了。\par
\begin{align}
\mu_d&=\frac{1}{\mu_0}\frac{dB}{dH_1}\\
&=\frac{1}{\mu_0}\left(a-\frac{3}{2}bH_m^2+\frac{3}{2}bH_m^2\cos (2\omega t)\right)
\end{align}
\reposubsec{行内公式}
刚才的两种公式都是行间公式,每次进入公式环境,LaTeX都会自动新起一行。但如果想要在文字的中间插入公式或特殊符号,就需要行内公式了。很简单,用两个美元符号,将要写的公式放在中间即可,比如$\$\backslash mu\_d=\backslash frac\{1\}\{\backslash mu\_0\}\backslash frac\{dB\}\{dH\_1\}\$$可以表示$\mu_d=\frac{1}{\mu_0}\frac{dB}{dH_1}$作为行内公式。作为行内公式,latex真的比word排版得好看的多。